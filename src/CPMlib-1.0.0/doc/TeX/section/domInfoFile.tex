\section{領域分割情報ファイルの仕様}
\label{domInfoFile}

以下に,領域分割情報ファイルの仕様とサンプルを示します.

以下の例では,活性サブドメイン数が20ですので,実行時のMPIプロセス数は
20以上である必要があります.

\begin{spacing}{0.65}
\begin{itembox}[l]{領域分割情報ファイルの仕様}
{\tt
\begin{verbatim}
"DomainInfo"
{
  "G_org"    = (1.0  , 2.0  , 3.0  ) //全計算領域の基点(実数,必須)
  "G_voxel"  = (20   , 24   , 16   ) //格子分割数(整数,必須)
  "G_region" = (100.0, 80.0 , 50.0 ) //計算領域の大きさ(実数,G_pitchと排他)
  "G_pitch"  = (1.0  , 0.8  , 0.5  ) //格子分割幅(実数,G_regionと排他)
  "G_div"    = (2    , 3    , 4    ) //全計算領域の分割数(整数)
  /* regionとpitchの両方が指定されているときはpitchを使用
     格子数が分割数で割り切れないときは,あまりを先頭から1ずつ分配する
   */
}
"ActiveSubDomains" //活性サブドメイン情報
{
  //pos =(ni,nj,nk)
  //bcid=(mx,my,mz,px,py,pz)

  //nk=0
  "domain[@]" { "pos"=(0,0,0) "bcid"=(1,3,5,0,0,0) }
  "domain[@]" { "pos"=(1,0,0) "bcid"=(0,3,5,2,0,0) }
  "domain[@]" { "pos"=(0,1,0) "bcid"=(1,0,5,0,0,0) }
  "domain[@]" { "pos"=(1,1,0) "bcid"=(0,0,5,2,0,0) }
  "domain[@]" { "pos"=(0,2,0) "bcid"=(1,0,5,0,4,0) }
  "domain[@]" { "pos"=(1,2,0) "bcid"=(0,0,5,2,4,0) }

  //nk=1
  "domain[@]" { "pos"=(0,0,1) "bcid"=(1,3,0,0,0,8) }
  "domain[@]" { "pos"=(1,0,1) "bcid"=(0,3,0,2,0,8) }
  "domain[@]" { "pos"=(0,1,1) "bcid"=(1,0,0,0,0,0) }
  "domain[@]" { "pos"=(1,1,1) "bcid"=(0,0,0,2,0,0) }
  "domain[@]" { "pos"=(0,2,1) "bcid"=(1,0,0,0,4,0) }
  "domain[@]" { "pos"=(1,2,1) "bcid"=(0,0,0,2,4,0) }

  //nk=2
  "domain[@]" { "pos"=(0,1,2) "bcid"=(1,7,0,0,0,0) }
  "domain[@]" { "pos"=(1,1,2) "bcid"=(0,7,0,2,0,0) }
  "domain[@]" { "pos"=(0,2,2) "bcid"=(1,0,0,0,4,0) }
  "domain[@]" { "pos"=(1,2,2) "bcid"=(0,0,0,2,4,0) }

  //nk=3
  "domain[@]" { "pos"=(0,1,3) "bcid"=(1,7,0,0,0,6) }
  "domain[@]" { "pos"=(1,1,3) "bcid"=(0,7,0,2,0,6) }
  "domain[@]" { "pos"=(0,2,3) "bcid"=(1,0,0,0,4,6) }
  "domain[@]" { "pos"=(1,2,3) "bcid"=(0,0,0,2,4,6) }
}
\end{verbatim}
}
\end{itembox}
\end{spacing}

\clearpage
