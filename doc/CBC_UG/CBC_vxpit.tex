\begin{abstract}
本章では,パラメータ入力支援ツールV-Xpitを用いたXMLパラメータの入力について説明します.
V-XpitはJava実行環境で動作するGUIを備えたXMLパラメータ入力支援アプリケーションです.
入力パラメータのデータ構造を定義したファイルを指定し,それをひな形としてパラメータの設定,編集を行います.
要素とパラメータは,構造定義ファイルに記述された構造定義,記述ルール,データ型に従い入力します.
\end{abstract}
%
\graphicspath{{./fig_Vxpit/}}

\section{V-Xpitのインストール}
\label{sec:install V-Xpit}

JRE(Java Runtime Environment) version 6がインストルされていることを確認した後,V-Xpitのインストーラを起動,インストールを実行してください.
インストールについての詳細は,V-Xpitユーザガイドをご覧ください.

\section{パラメータ入力}
\label{sec:parameter input}

\subsection{V-Xpitを用いた入力パラメータの作成}
V-Xpitを用いたパラメータ入力は,ひな形となる構造定義ファイルが必要になります.ソースファイルに同梱の構造定義ファイル(CBC.xsd)を利用してください.
新規作成の場合には,「ファイル > 新規作成」から,既にあるXMLファイルを編集する場合には「ファイル > 開く...」からスタートしてください.

\subsection{入力要素}

\begin{enumerate}
\item 必須要素と任意要素
\item 固定値入力要素
\item リスト選択要素
\item 任意値入力要素
\item 依存・排他関係
\end{enumerate}


\subsection{新規パラメータの作成}



\subsection{既存パラメータファイルの修正}


検証