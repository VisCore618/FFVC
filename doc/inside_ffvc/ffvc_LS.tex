\graphicspath{{./fig_EBCS/}}
%
\section{圧力のPoisson方程式}

\subsection{反復処理}

FFVCで扱う圧力のPoisson方程式は,次のような形式となる.

\begin{equation}
\nabla \left( \nabla p^{\,n+1} \right) \,=\, \frac{1}{\Delta t} div\, ( \bm{u}^{\,*}) \,+\, div( \beta \bm{F} )
\label{eq:Poisson_ffvc}
\end{equation}


\noindent ここで,$\beta$は体積率,$\bm{F}$は外力ベクトルである.

ノイマン条件とディリクレ条件のフラグを導入し,
対角項が1となるようにスケーリングして,連立一次方程式$Ax = b$を得る.

\begin{equation}
\underbrace{ p  -  \frac{1}{\sum \limits_l \phi^{\,N}_{\,l}} \sum \limits_l {\left( p \,\phi^{\,D} \,\phi^{\,N} \right)}_{\,l} } \limits_{A x }
\,=\,
 \displaystyle { 
\underbrace{ -
\frac{1}{\sum \limits_l \phi^{\,N}_{\,l}} 
\left\{ 
\frac{\Delta x}{\Delta t} \,\sum \limits_l u^*_l \,+\, \Delta x \sum \limits_l (\beta F_l)
\right\}
} \limits_{ b }
}
\label{eq:ebcs_poisson-discrete}
\end{equation}

\noindent 右辺ベクトルの第一項めは境界条件も含む.


反復回数を$m$としてSOR法系の反復式を構成すると,

\begin{equation}
\left.
\begin{array}{lll}
\vspace{2mm}
\tilde{p} & = & \displaystyle {  b \, + \,
 \frac{1}{\sum \limits_l \left( \, \phi^{\,N}_{\,l} \, \right)} \sum \limits_l {\left( p^m \,\phi^{\,D} \,\phi^{\,N}\right)}_{\,l} 
 \,=\,
\frac{1}{\sum \limits_l \left( \, \phi^{\,N}_{\,l} \, \right)} 
\left \{ \,
\sum \limits_l {\left( p^m \,\phi^{\,D} \,\phi^{\,N}\right)}_{\,l} 
- \left( h^2 \,\psi \,+\,  \psi^{BC} \, \right)
\, \right \} } \\
\vspace{2mm}
\Delta p & = & \tilde{p} \,-\, p^{\,m} \\
\vspace{5mm}
p^{\,m+1} & = & p^{\,m} \,+\, \omega\, \,\Delta p \\

\end{array} \qquad \right \}
\label{eq:poisson_iteration}
\end{equation}



\begin{algorithm}
\caption{Iterative procedure in Fractional step method.}
\label{algo:FS}
\begin{algorithmic}

\State Increment state : $\bm{u}^{\,n} \gets  \bm{u}^{\,n-1}, \quad p^{\,n} \gets  p^{\,n-1}$

\State Evaluate space term : $\displaystyle{ \bm{w} \gets \frac{1}{Re} \frac{\partial^2 \bm{u}^{\,n}}{\partial x^2} \,-\, \frac{\partial}{\partial x} (\bm{u} \bm{u})^{\,n} }$

\State Time integration : $\bm{u}^* \,=\, \bm{u}^n \,+\, \Delta t \, \bm{w}$

\State External force : $\bm{u}^* += \Delta t \bm{f}$

\State Boundary condition for $\bm{u}^*$

\State Set initial value for iteration : $\bm{u}^{(0)} \,\gets \, \bm{u}^*$

\State Calculate non-iterative source : $ \displaystyle{ \bm{s}_0 \gets \frac{\Delta x}{\Delta t} \sum \limits_l u^*_l }$

\State Add iterative source and set initial value : $\bm{s}_1^{(0)} \gets \bm{s}_0 + \Delta x \sum \limits_l (\beta F^n_l), \quad \bm{F}^n = f(\bm{u}^{\,n})$

\State RHS vector of Poisson : $ \displaystyle{ \bm{b}^{(0)} \gets - \frac{1}{\sum \limits_l \phi^{\,N}_{\,l}} \left( \bm{s}_1^{(0)} \right) \,\equiv\,
- \frac{1}{\sum \limits_l \phi^{\,N}_{\,l}} 
\left\{ \frac{\Delta x}{\Delta t} \,\sum \limits_l u^*_l \,+\, \Delta x \sum \limits_l (\beta F^n_l) \right\} }$

\State Calculate $L_2$ norm of $\bm{b}^{(0)}$ : $\| \bm{b}^{(0)} \|_2$

\State Calculate initial residual : $ \bm{r}^{(0)} \gets \bm{b}^{(0)} -A \bm{x}^{(0)}, \quad \bm{x}^{(0)} = \{p^n_0,\,p^n_1,\,...,p^n_{imax \times jmax \times kmax}\}^T$

\State Take $L_2$ norm of $\bm{r}^{(0)}$ : $\| \bm{r}^{(0)} \|_2$

\For {$m=0$ to IterationMax}

\State Solve $A \bm{x}^{(m+1)} = \bm{b}^{(m)}$

\State Update velocity : $\bm{u}^{(m+1)} \gets \bm{u}^* - \Delta t \,\nabla p^{(m+1)}$

\State Calculate divergence : $\displaystyle{ \bm{dv} \gets \frac{\Delta x}{\Delta t} \sum \limits_l u^{m+1}_l }$

\State Calculate source term : $\bm{s}_1^{(m+1)} \gets \bm{s}_0 + \Delta x \sum \limits_l \left(\beta F^{(m+1)}_l \right), \quad \bm{F}^{(m+1)} = f(\bm{u}^{(m+1)})$

\State Calculate RHS vector : $ \displaystyle{ \bm{b}^{(m+1)} \gets - \frac{1}{\sum \limits_l \phi^{\,N}_{\,l}} \left( \bm{s}_1^{(m+1)} \right)}$

\If {$\displaystyle{ (\frac{\Delta t}{\Delta x} \times \bm{dv} < \varepsilon) }$}
\State Break Loop
\EndIf

\EndFor

\end{algorithmic}
\end{algorithm}




%
\subsection{ノルム}
ノルムの定義は,下記とする.
\begin{equation}
\begin{array}{ll}
L_2 &
\displaystyle{ \| x \|_p \,=\, 
\left( \sum_{p=0}^n {|x_i|^p} \right)^{\frac{1}{p}}
}\\
\vspace{2mm}
Maximum & \| x \|_{\infty} \,=\, max|x_i|\\
\end{array}
\label{eq:def_norm}
\end{equation}

利用可能なノルムは,次の3種類である.

\begin{equation}
\begin{array}{lc} \toprule
Type & Norm \\ \midrule
\vspace{2mm}
\Delta x / b & \displaystyle{ \frac{\|x^{m+1}-x^m \|_2}{\|b\|_2} } \\ 
\vspace{2mm}
r/b & \displaystyle{ \frac{\| r \|_2}{\|b\|_2} } \\ 
\vspace{2mm}
r/r_0 & \displaystyle{ \frac{\|r\|_2}{\|r_0\|_2} } \\  \bottomrule
\end{array}
\label{eq:iter_norm}
\end{equation}






%
\section{GMRES}

GMRES(Generalized Minimum Residual)は,大型疎行列を係数行列にもつ連立一次方程式$A x = b$の解法アルゴリズムで,Krylov部分空間法に分類される.


m次のKrylov部分空間は
\begin{equation}
\displaystyle{ K_m \,=\, span \{ v^{(1)},\, Av^{(1)},\, A^2 v^{(1)},\, ...,\, A^{m-1} v^{(1)} \} }
\label{eq:subspace}
\end{equation}

\noindent で,$v^{(1)}=r^{(0)}/\|r^{(0)}\|$である\footnote{上付きの$(i)$を反復,下付きの$i$を要素として表記する.}.
GMRES法は,Krylov部分空間の直交基底ベクトル列をArnoldi-modified Gram-Schmidt法により構築し,
$x^{(0)}+K_m$の形式の全ベクトルの残差のユークリッドノルムを最小化する.
各反復毎に残差ノルムが最小になるので,残差ノルムは単調に減少する特徴をもつ.

GMRESアルゴリズムを適用するためには,係数行列$A$は正則である必要があるが,非対称($A \neq A^T$)でもよい.
Arnoldi過程に基づく長い漸化式を用いるので,GMRES法は反復回数の増加に伴い,1反復あたりの演算量と記憶容量が増大する問題点がある.このため,実用上「リスタート」が併用される.リスタートがない場合,GMRES法はnステップ($n \times n$の行列サイズの場合)で収束が保証されるが,多大な記憶領域と演算量となるので,リスタートは必須である.

\vspace{3mm}
GMRESのアルゴリズムはmodified Gram-Schmidt直交化を用いるので,直交基底は次のようになる\cite{bahi:07}.

\begin{algorithm}
\caption{Orthogonal basis function. Count i means a number of iteration.}
\label{algo:GS-ortho}
\begin{algorithmic}

\State $w^{(i)} \,\gets \, A v^{(i)}$
\For {$k=1$ to $i$}
\State $w^{(i)} \,\gets \, w^{(i)} - (w^{(i)}, v^{(k)})\, v^{(k)}$
\EndFor
\State $v^{(i+1)} \,\gets \, w^{(i)} \,/\, \|w^{(i)}\|_2$

\end{algorithmic}
\end{algorithm}


GMRESアルゴリズムは,残差ノルムを反復なしに計算することができる.
十分に正確になるまで一連の残差ノルムを計算し,それから時間のかかる反復を行う効率的な計算ができる.
$y_k$は残差ノルムを最小化するベクトルに選ぶと,反復過程は次式になる.

\begin{equation}
x^{(i)} \,=\, x^{(0)} \,+\, y_1 v^{(1)} \,+\, y_2 v^{(2)} \,+\, \dots \,+\, y_i v^{(i)}
\label{eq:GMRES iteration}
\end{equation}


アルゴリズム\ref{algo:GS-ortho}に示すように,GMRES法は反復回数の増加に従い,ベクトル$w^{(i)}$の記憶容量が増加し,演算量も線形に増える.この問題点を解消するためにリスタートが用いられる.リスタートではリスタート周期(Frequency of restart)としてm回の反復数を選び,反復をm回で打ち切る.次のm回の反復の初期値として前回の中間値を用いる.
mの選択は一意ではなく,少なすぎると収束が遅くなり,多すぎると記憶容量と演算量が過大になる.

\vspace{3mm}
GMRESのアルゴリズムは,次のようになる.
\begin{enumerate}
\item QR分解を用いたHessenberg行列$H$と直交基底$V$を与えるArnoldi過程を計算する.
\item 残差の最小化問題をHessenberg行列を用いて解く際にQR分解を行うが,それにはGivens回転を計算する.$H$の前の要素にGivens回転を適用し,次の新しい回転を計算,最後に最小化問題の右辺ベクトル$Z$に適用する.
\item 必要であれば,解ベクトル$X$を更新するため,$H$と$Z$を用いて最小化問題を解く.
\end{enumerate}



FGMRES(Flexible GMRES)法は,収束性を高めるために前処理を用いる方法である.前処理にはSORやGMRES自身など様々な反復法が利用できる.



%
\begin{algorithm}
\caption{FGMRES algorithm}
\label{algo:fgmres}

\begin{algorithmic}
\State m             : Number of iterations between each restart (Frequency of restart)
\State size          : Size of a matrix
\State Iter\_Max     : Upper limit of GMRES iteration
\State x(size)       : Solution vector
\State b(size)       : right hand side vector
\State w(size)       : Intermediate vector used in the Arnoldi's process to compute H
\State r(size)       : Residual vector r = b - Ax
\State rm(m+1)       : Residual vector of the minimization problem
\State c(m+1)        : Cosine for Givens rotations
\State s(m+1)        : Sine for Givens rotations
\State h(m+1, m+1)   : Matrix for the Hessenberg decomposition
\State v(size, m+1)  : Orthogonal basis for the Krylov subspace
\State z(size, m+1)  : Right-hand side vector for the residual minimization problem
\State K             : Approximately inverse matrix of A
\State $\epsilon$    : Convergence criteria
\State $\varepsilon$ : Convergence criteria
\State 

\State Initialize: $w,c,s,r,h,v,z \gets 0$
\State $x \gets x_0$
\State

% outer loop
\For {$j=1$ to Iter\_Max}

\State $r \gets b - Ax$
\State $\beta \gets \|r\|$

% 収束のチェック
\If {($\beta < \epsilon \|r\|$)}
\State \bf{return}
\EndIf

% 残差ベクトル列
\State $rm \gets \{\beta,\,0,\,...,\,0\}^T$
\State $v^{(1)} \gets r / \beta$
\State

% inner loop
\For {$i=1$ to $m$}
\State $z^{(i)} \gets K^{-1}_{i,\,j}\, v^{(i)} \quad or \quad z^{(i)} \gets v^{(i)}\,(without\, preconditioning)$
\State $w \gets A z^{(i)}$

\For {$k=1$ to $i$}
\State $h_{k,\,i} \gets (w,\,v^{(k)})$
\State $w \gets w - h_{k,\,i}\,v^{(k)}$
\EndFor

\State $h_{i+1,\,i} \gets \|w\|$
\State

\If {($h_{i+1,\,i} < \epsilon$)}
\State Warning: pseudo-converge
\State $n \gets i-1$
\State goto jump\_1
\EndIf
\State

\State $v^{(i+1)} \gets w / h_{i+1,\,i}$

\algstore{fgmres}
\end{algorithmic}

\end{algorithm}

%
\clearpage

%
\begin{algorithm}
\caption{FGMRES algorithm (continued)}

\begin{algorithmic}

\algrestore{fgmres}
\State 

\For {$k=2$ to $i$}
\State $ht \gets h_{k-1,\,i}$
\State $h_{k-1,\,i}\, \gets \,  c_{k-1} ht \,+\, s_{k-1} h_{k,\,i}$
\State $h_{k,\,i}\,   \gets \, -s_{k-1} ht \,+\, c_{k-1} h_{k,\,i}$
\EndFor

\State
\State $\delta \gets \sqrt{h_{i,\,i}^2 + h_{i+1,\,i}^2}$

\State
\If {($\delta < \epsilon$)}
\State $n \gets i-1$
\State goto jump\_1
\EndIf
\State

\State $c_i \gets h_{i,\,i} / \delta$
\State $s_i \gets h_{i+1,\,i} / \delta$
\State $h_{i,\,i} \gets c_i h_{i,\,i} + s_i h_{i+1,\,i}$
\State

\State $rm_{i+1} \gets -s_i r_i$
\State $rm_{i} \gets c_i r_i$
\State $\rho \gets |rm_{i+1}|$
\State

\If {($\rho < \varepsilon \|b\|$)}
\State $n \gets i$
\State goto jump\_1
\EndIf

\EndFor

\State
\State $n \gets m$

\State
\State \bf{jump\_1:}
\State


\For {$j=n$ downto 2}
\State $rm_j \gets rm_j / h_{j,\,j}$

\For {$k=1$ to $j-1$}
\State $rm_k \gets rm_k - rm_j \, h_{k,\,j}$
\EndFor

\EndFor

\State
\State $rm_1 \gets rm_1 / h_{1,\,1}$
\State

\For {$i=1$ to $n$}
\State $x \gets x + rm_i \, z^{(i)}$
\EndFor

\EndFor

\end{algorithmic}

\end{algorithm}















