
\section{コンパイルエラー}
\label{sec:compile_error}

%
\subsection{VMware上のFedora 14でのコンパイル}
\label{VMware}

\begin{indentation}{3zw}{0zw}
VMware上でFedrora14 32bitがゲストOSの場合,V-Sphere Ver. 1.8.4のコンパイルがうまくいかない場合がある.
その場合,以下の手順で成功する事例がある.

\begin{program}
$ aclocal
$ autoconf
$ automake -a
$ configure [option...]
$ make
\end{program}
\end{indentation}

\subsection{Fedoraにmpich2をインストールした場合のトラブル}
\begin{indentation}{3zw}{0zw}
Linux OSがFedoraで,並列ライブラリとしてmpich2をインストールした場合,
V-Sphereのインストールでリンクエラーとなる場合がある.

その場合,以下のいずれかの手順で成功する事例がある.

\subsubsection{方法1}
mpich2のコンパイララッパーを使用する.

mpich2は、configure時に指定したコンパイラをラップした
コンパイルコマンドがbin配下にインストールされる.

mpich2のインストールディレクトリが/usr/local/mpich2/の場合,/usr/local/mpich2/bin配下に
mpif77,mpif90,mpicc,mpic++等のコマンドが格納されている.
これらのコマンドは、mpich2がintelコンパイラでmakeされて
いれば、内部的にintelコンパイラがコールされ,かつ,MPI
プログラムに必要なライブラリを自動的にリンクしてくれる.

そこで、V-Sphereのconfigure時に,\\
\hspace{1cm}CC=icc\\
\hspace{1cm}CXX=icpc\\
\hspace{1cm}FC=ifort\\
\hspace{1cm}F90=ifort\\
の代わりに,\\
\hspace{1cm}CC=/usr/local/mpich2/bin/mpicc\\
\hspace{1cm}CXX=/usr/local/mpich2/bin/mpic++\\
\hspace{1cm}FC=/usr/local/mpich2/bin/mpif77\\
\hspace{1cm}F90=/usr/local/mpich2/bin/mpif90\\
をそれぞれ指定すると,リンクエラーは解消される.

\subsubsection{方法2}
不足しているライブラリを強制的にリンクする.

mpich2の場合、libmplとlibpthreadをリンクする必要が
あるので,これらのリンク指示をMakefile等に
追記することで,リンクエラーを回避する.

以下のファイルを編集する必要がある.

\paragraph{V-Sphereコンパイル時}
configure実行後に,src/utility/sphDataGather/Makefileの
以下を修正する.\\
218、219行目の「-lmpich」の記述の後に「-lmpl -lpthread」を
追加\\
例)\\
dataGather\_LDADD = -lsphcfg -lmpich -lmpl -lpthread -L/usr/local/mpich2-install/lib -lxml2 -lz -lm\\
sphDataGather\_LDADD = -lsphcfg -lmpich -lmpl -lpthread -L/usr/local/mpich2-install/lib -lxml2 -lz -lm

\paragraph{ソルバープロジェクト(CBC)}
project\_local\_settingsの「LIBS=」の行に「-lmpl -lpthread」を追加\\
(例)\\
LIBS=-lifport -lifcore -lmpl -lpthread

\end{indentation}

\subsection{FedoraにOpenMPIをインストールする場合のトラブル}
\subsubsection{OpenMPIのコンフィギュア}
\begin{indentation}{3zw}{0zw}
Linux OSがFedoraで,並列ライブラリとしてOpenMPIのインストールがうまくいかない場合,OpenMPIのconfigure時に,
\begin{program}
--enable-contrib-no-build=vt
\end{program}
オプションを付加すると,成功する事例がある.
\end{indentation}

\subsubsection{OpenMPIのインストール}
\begin{indentation}{3zw}{0zw}
Linux OSがFedoraで,並列ライブラリとしてOpenMPIをルート権限が必要な場所に
\begin{program}
$ sudo make install
\end{program}
でインストールする場合,
\begin{program}
icc: command not found
\end{program}
というインストールエラーが出る場合がある.\\

その場合,以下の手順で成功する事例がある.\\

rootユーザの.bashrcや.bash\_profielにintelコンパイラのPATHなどの環境設定を記述する.その後,rootユーザとして,
\begin{program}
# make install
\end{program}
\end{indentation}